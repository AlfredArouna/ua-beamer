\section{Colors}

\subsection{Background on the official colors}

\begin{note}
If you don't care about the technical details, you can skip this and directly
go to subsection~\ref{subsection:usingthecolors}.
\end{note}

The two official colors of the UA, a blue and a red tone, are originally given
in PMS format, a \emph{proprietary} format issued by the Pantone Inc.\
corporation. One particular feature of the PMS color format is that it is, as
opposed to RGB or CMYK, \emph{device independent}. This means that the
definition of the color does not consist of instructions such as \enquote{put
  37\% red, 12\% green, and 45\% blue and mix it all together}, but really is a
  concise description of what the final result, be it printed or displayed,
  physically looks like on the medium under determined ambient light
  conditions.

The PMS format is richer than RGB in the sense that it can embrace fluorescence
effects, gold or silver shine, special coatings (matte and brilliant), in
general everything that has to do with the actual appearance of the color on
the medium. It comes thus as no surprise that there is no (exact) mapping
between the PMS and RGB color spaces; more specifically: the mapping -- if it
exists -- is device-dependent.

The \emph{only} way to get a perfect PMS 1955 red, for example, is to have a
the plot file ready with the (proprietary) PMS color information in it, and
have it printed on a PMS-ready printer which needs to be filled the special PMS
1955 ink beforehand. This process is eponymous for colors which are not
composed of different types of (yellow, blue, red) inks: they are called
\emph{single-spot colors}, and most Pantone\textsuperscript{\textregistered}
colors are of such kind.


\subsection{Conversion to RGB/CMYK}

When the computer screen or any other non-PMS-ready medium is the primary
output source, having such rigorous rules may be rather obstructive. To
overcome such restrictions, many vendors of graphical software try to translate
PMS into RGB/CMYK by making use of special monitor color calibration data
(e.g., the ICC color profiles, \lstinline!.icc!). Unfortunately, if files
containing PMS color information are displayed (printed) with programs
(printers) which do not support PMS mechanisms (which is most often the case in
non-professional environments), the color output will look disturbed. On top of
that, and surprisingly for the novice, the output will look disturbed in
different ways on different screens (printers) because of the inherent
device-dependence of PMS.

%To overcome the undesired side-effects of working with the official UA
%corporate design in a non-PMS environment (like most beamers and computer
%screens should provide),
That is why this theme aims to use non-PMS versions of the UA's official
colors, including modified versions of the logos.

\begin{table}
\centering
\newlength{\boxsize}
\setlength{\boxsize}{2cm}

\begin{tabular}{cccc}
\definecolor{uablue11}{RGB}{  0, 61,100}
\colorbox{uablue11}{\rule{0mm}{\boxsize}\rule{\boxsize}{0mm}}
&
\definecolor{uablue12}{cmyk}{1,0.3,0,0.62}
\colorbox{uablue12}{\rule{0mm}{\boxsize}\rule{\boxsize}{0mm}}
&
\definecolor{uablue13}{RGB}{0,96,128}
\colorbox{uablue13}{\rule{0mm}{\boxsize}\rule{\boxsize}{0mm}}
&
\definecolor{uablue14}{cmyk}{1,0.25,0,0.5}
\colorbox{uablue14}{\rule{0mm}{\boxsize}\rule{\boxsize}{0mm}}\\[2mm]
\definecolor{uared11}{RGB}{126,  0, 47}
\colorbox{uared11}{\rule{0mm}{\boxsize}\rule{\boxsize}{0mm}}
&
\definecolor{uared11}{cmyk}{0,1,0.54,0.46}
\colorbox{uared11}{\rule{0mm}{\boxsize}\rule{\boxsize}{0mm}}
&
\definecolor{uared11}{RGB}{161,0,64}
\colorbox{uared11}{\rule{0mm}{\boxsize}\rule{\boxsize}{0mm}}
&
\definecolor{uared11}{cmyk}{0,1,0.6,0.37}
\colorbox{uared11}{\rule{0mm}{\boxsize}\rule{\boxsize}{0mm}}
\end{tabular}

\caption{PMS 302 and PMS 1955 translated to CMYK/RGB in different ways. Left to
right: UA website \cite{KAN::} RGB, UA website \cite{KAN::} CMYK, conversion
chart~\cite{::TDC} RGB, conversion chart~\cite{::TDC} \@CMYK. To determine which
version is closest to the actual colors PMS 1955 and PMS 302 on your
monitor/printer/beamer, you would need a physical color sample (such as
provided on the Pantone\textsuperscript{\textregistered} charts) to compare.}
\label{table:reds-blues}
\end{table}

Now, as explained above there is no device-independent conversion between the
original PMS directives and the CMYK color space; several tables exist which
are valid for different work environments. On the official sites of the UA
\cite{KAN::}, you'll find particular RGB and CMYK values, other resources,
e.g., \cite{::TDC}, provide other numbers (see table~\ref{table:reds-blues}).

For the sake of consistency between RGB and CMYK values, the \texttt{beamer}
theme uses the CMYK values for PMS 302 and PMS 1955 provided on \cite{::TDC}
for reference (see table~\ref{table:reds-blues}, last column, and
table~\ref{table:allcolors}).


\subsection{Using the colors}\label{subsection:usingthecolors}

Getting a consistent look-and-feel throughout your presentations requires
sticking to a particular style scheme, most of which is being implemented in
the \texttt{beamer} style file already. One particular aspect, though, can only
be controlled by the user, and that is the colors that are used in the running
text, tables, and figures.

Although essentially consistent of only two colors, within \texttt{beamer}
referred to as \lstinline!uablue! and \lstinline!uared!, provide more diversity
than one might expect and should be  \emph{exclusively} used  in all the
slides. The user should be aware that this directive includes \emph{tables,
figures, and graphics of most kinds}. For an example of usage see
figure~\ref{fig:coloredgraphs}.

\begin{figure}
\setlength{\figurewidth}{10cm}
\setlength{\figureheight}{5cm}
\centering
\begin{tikzpicture}

% Axis at [0.13 0.11 0.78 0.81]
\begin{axis}[
axis on top,
scale only axis,
width=\figurewidth,
height=\figureheight,
xmin=0, xmax=10,
ymin=0, ymax=7
]

\addplot [
color=uablue,
solid,
line width=1pt
] coordinates{
 (0,5.23398) (1,4.38268) (2,4.69238) (3,4.32081) (4,3.54144) (5,3.72536) (6,3.16156) (7,2.59604) (8,2.20231) (9,2.60195) (10,1.60575)
};

\addplot [
color=uared,
solid,
line width=1pt
] coordinates{
 (0,2.02709) (1,2.02644) (2,1.68711) (3,1.84761) (4,1.78782) (5,1.24044) (6,1.26026) (7,0.992143) (8,0.859557) (9,0.689706) (10,0.36891)
};

\addplot [
color=uablue50,
solid,
line width=1pt
] coordinates{
 (0,3.2703) (1,3.30947) (2,3.74723) (3,3.45218) (4,3.3806) (5,3.13813) (6,2.81847) (7,2.79316) (8,2.64385) (9,3.01041) (10,2.72024)
};

\addplot [
color=uared50,
solid,
line width=1pt
] coordinates{
 (0,5.49163) (1,4.97712) (2,5.031) (3,5.42669) (4,5.41402) (5,5.4996) (6,5.311) (7,5.34227) (8,5.46331) (9,5.46677) (10,5.12875)
};

\addplot [
color=uablue25,
solid,
line width=1pt
] coordinates{
 (0,5.56199) (1,5.2627) (2,5.27883) (3,5.38474) (4,5.90686) (5,6.11345) (6,6.04402) (7,6.36515) (8,6.62563) (9,6.38024) (10,6.33876)
};

\addplot [
color=uared25,
solid,
line width=1pt
] coordinates{
 (0,3.7947) (1,4.08811) (2,4.28361) (3,4.45264) (4,4.46625) (5,5.28088) (6,5.32715) (7,5.54452) (8,5.439) (9,5.99242) (10,6.35941)
};

\addplot [
color=vividbrown,
solid,
line width=1pt
] coordinates{
 (0,1.43939) (1,1.51392) (2,2.30537) (3,2.05355) (4,3.05479) (5,2.9632) (6,3.16279) (7,3.3234) (8,4.15725) (9,4.278) (10,4.78022)
};

\end{axis}

\end{tikzpicture}

\caption{Example usage of the official colors within a set of graphs, using
also the exception color.}
\label{fig:coloredgraphs}
\end{figure}

If you need to emphasize a particular aspect in your slides (graphs, tables),
you can (within \texttt{beamer}) use the \lstinline!\alert{}! macro (e.g.,
\lstinline!\alert{This is alerted text.}!). For the situation where something
needs to stick out in a pie chart, for example, where the ordinary colors (UA
blue and red) have been used up already, a third color has been added (see
table~\ref{table:allcolors}). It is to be used scarcely and strictly for
highlighting purposes (see, for example, figure~\ref{fig:coloredgraphs}).


\begin{table}
\setlength{\figurewidth}{12mm}
\centering
\begin{tabulary}{\textwidth}{LCCC}\toprule
Color sample & \colorsquare{\figurewidth}{uablue100} & \colorsquare{\figurewidth}{uablue75} & \colorsquare{\figurewidth}{uablue50}\\
CMYK & (100, 30, 0, 62) & (75, 23, 0, 47) & (50, 15, 0, 31)\\
RGB  & (0, 61, 100)     & (40, 128, 158)  & (96, 166, 191)\\
\texttt{beamer} name & \lstinline!uablue100! & \lstinline!uablue75! & \lstinline!uablue50!\\[3mm]
Color sample & \colorsquare{\figurewidth}{uablue25} & \colorsquare{\figurewidth}{uablue10} & \colorsquare{\figurewidth}{uablue5}\\
CMYK & (25, 8, 0, 16)  & (10, 3, 0, 6)   & (5, 2, 0, 3)\\
RGB  & (166, 209, 222) & (218, 235, 242) & (235, 245, 247)\\
\texttt{beamer} name & \lstinline!uablue25! & \lstinline!uablue10! & \lstinline!uablue5!\\\midrule

Color sample & \colorsquare{\figurewidth}{uared100} & \colorsquare{\figurewidth}{uared75} & \colorsquare{\figurewidth}{uared50}\\
CMYK & (0, 100, 54, 46) & (0, 75, 41, 35) & (0, 50, 27, 23)\\
RGB  & (161, 0, 64)     & (184, 46, 101)  & (207, 103, 145)\\
\texttt{beamer} name & \lstinline!uared100! & \lstinline!uared75! & \lstinline!uared50!\\[3mm]
Color sample & \colorsquare{\figurewidth}{uared25} & \colorsquare{\figurewidth}{uared10} & \colorsquare{\figurewidth}{uared5}\\
CMYK & (0, 25, 14, 12)  & (0, 10, 5, 5)   & (0, 5, 3, 2)\\
RGB  & (232, 174, 197) & (245, 220, 230) & (250, 237, 242)\\
\texttt{beamer} name & \lstinline!uared25! & \lstinline!uared10! & \lstinline!uared5!\\\midrule

Color sample & \colorsquare{\figurewidth}{vividbrown} & \colorsquare{\figurewidth}{vividbrown75} & \colorsquare{\figurewidth}{vividbrown50}\\
CMYK & (0, 28, 67, 16) & (0, 20, 48, 12) & (0, 13, 31, 8)\\
RGB  & (215, 154, 70) & (225, 179, 116)  & (235, 205, 163)\\
\texttt{beamer} name & \lstinline!vividbrown100! & \lstinline!vividbrown75! & \lstinline!vividbrown50!\\[3mm]
Color sample & \colorsquare{\figurewidth}{vividbrown25} & \colorsquare{\figurewidth}{vividbrown10} & \colorsquare{\figurewidth}{vividbrown5}\\
CMYK & (0, 6, 15, 4) & (0, 2, 6, 2) & (0, 1, 3, 1)\\
RGB  & (245, 230, 209)  & (251, 245, 237) & (253, 250, 246)\\
\texttt{beamer} name & \lstinline!vividbrown25! & \lstinline!vividbrown10! & \lstinline!vividbrown5!\\\bottomrule
\end{tabulary}

\caption{Overview over the two main colors used in theme as well as the third color for exceptions and highlighting.}
\label{table:allcolors}
\end{table}

%\begin{table}
%\begin{tabular*}{\textwidth}{cc}
%\toprule
%Category 1 &
%\colorbox{uablue100}{\rule{0mm}{5mm}\rule{5mm}{0mm}}
%\colorbox{uablue75}{\rule{0mm}{5mm}\rule{5mm}{0mm}}
%\colorbox{uablue50}{\rule{0mm}{5mm}\rule{5mm}{0mm}}
%\colorbox{uablue25}{\rule{0mm}{5mm}\rule{5mm}{0mm}}
%\colorbox{uablue10}{\rule{0mm}{5mm}\rule{5mm}{0mm}}
%\colorbox{uablue5}{\rule{0mm}{5mm}\rule{5mm}{0mm}}\\
%&
%\colorbox{uared100}{\rule{0mm}{5mm}\rule{5mm}{0mm}}
%\colorbox{uared75}{\rule{0mm}{5mm}\rule{5mm}{0mm}}
%\colorbox{uared50}{\rule{0mm}{5mm}\rule{5mm}{0mm}}
%\colorbox{uared25}{\rule{0mm}{5mm}\rule{5mm}{0mm}}
%\colorbox{uared10}{\rule{0mm}{5mm}\rule{5mm}{0mm}}
%\colorbox{uared5}{\rule{0mm}{5mm}\rule{5mm}{0mm}}\\\midrule
%Category 2 &
%\colorbox{col22}{\rule{0mm}{5mm}\rule{5mm}{0mm}}
%\colorbox{col22!75!white}{\rule{0mm}{5mm}\rule{5mm}{0mm}}
%\colorbox{col22!50!white}{\rule{0mm}{5mm}\rule{5mm}{0mm}}
%\colorbox{col22!25!white}{\rule{0mm}{5mm}\rule{5mm}{0mm}}
%\colorbox{col22!10!white}{\rule{0mm}{5mm}\rule{5mm}{0mm}}
%\colorbox{col22!5!white}{\rule{0mm}{5mm}\rule{5mm}{0mm}}\\
%&
%\colorbox{col24}{\rule{0mm}{5mm}\rule{5mm}{0mm}}
%\colorbox{col24!75!white}{\rule{0mm}{5mm}\rule{5mm}{0mm}}
%\colorbox{col24!50!white}{\rule{0mm}{5mm}\rule{5mm}{0mm}}
%\colorbox{col24!25!white}{\rule{0mm}{5mm}\rule{5mm}{0mm}}
%\colorbox{col24!10!white}{\rule{0mm}{5mm}\rule{5mm}{0mm}}
%\colorbox{col24!5!white}{\rule{0mm}{5mm}\rule{5mm}{0mm}}\\
%&
%\colorbox{col21}{\rule{0mm}{5mm}\rule{5mm}{0mm}}
%\colorbox{col21!75!white}{\rule{0mm}{5mm}\rule{5mm}{0mm}}
%\colorbox{col21!50!white}{\rule{0mm}{5mm}\rule{5mm}{0mm}}
%\colorbox{col21!25!white}{\rule{0mm}{5mm}\rule{5mm}{0mm}}
%\colorbox{col21!10!white}{\rule{0mm}{5mm}\rule{5mm}{0mm}}
%\colorbox{col21!5!white}{\rule{0mm}{5mm}\rule{5mm}{0mm}}\\
%&
%\colorbox{col42}{\rule{0mm}{5mm}\rule{5mm}{0mm}}
%\colorbox{col42!75!white}{\rule{0mm}{5mm}\rule{5mm}{0mm}}
%\colorbox{col42!50!white}{\rule{0mm}{5mm}\rule{5mm}{0mm}}
%\colorbox{col42!25!white}{\rule{0mm}{5mm}\rule{5mm}{0mm}}
%\colorbox{col42!10!white}{\rule{0mm}{5mm}\rule{5mm}{0mm}}
%\colorbox{col42!5!white}{\rule{0mm}{5mm}\rule{5mm}{0mm}}\\\midrule
%Alert color &
%\colorbox{col32}{\rule{0mm}{5mm}\rule{5mm}{0mm}}
%\colorbox{col32!75!white}{\rule{0mm}{5mm}\rule{5mm}{0mm}}
%\colorbox{col32!50!white}{\rule{0mm}{5mm}\rule{5mm}{0mm}}
%\colorbox{col32!25!white}{\rule{0mm}{5mm}\rule{5mm}{0mm}}
%\colorbox{col32!10!white}{\rule{0mm}{5mm}\rule{5mm}{0mm}}
%\colorbox{col32!5!white}{\rule{0mm}{5mm}\rule{5mm}{0mm}}\\\bottomrule
%\end{tabular*}
%\caption{Color table.}
%\end{table}

%\colorbox{col11}{\rule{0mm}{5mm}\rule{5mm}{0mm}}
%\colorbox{col12}{\rule{0mm}{5mm}\rule{5mm}{0mm}}
%\colorbox{col13}{\rule{0mm}{5mm}\rule{5mm}{0mm}}
%\colorbox{col14}{\rule{0mm}{5mm}\rule{5mm}{0mm}}\\
%\colorbox{col21}{\rule{0mm}{5mm}\rule{5mm}{0mm}}
%\colorbox{col22}{\rule{0mm}{5mm}\rule{5mm}{0mm}}
%\colorbox{col23}{\rule{0mm}{5mm}\rule{5mm}{0mm}}
%\colorbox{col24}{\rule{0mm}{5mm}\rule{5mm}{0mm}}\\
%\colorbox{col31}{\rule{0mm}{5mm}\rule{5mm}{0mm}}
%\colorbox{col32}{\rule{0mm}{5mm}\rule{5mm}{0mm}}
%\colorbox{col33}{\rule{0mm}{5mm}\rule{5mm}{0mm}}
%\colorbox{col34}{\rule{0mm}{5mm}\rule{5mm}{0mm}}\\
%\colorbox{col41}{\rule{0mm}{5mm}\rule{5mm}{0mm}}
%\colorbox{col42}{\rule{0mm}{5mm}\rule{5mm}{0mm}}
%\colorbox{col43}{\rule{0mm}{5mm}\rule{5mm}{0mm}}
%\colorbox{col44}{\rule{0mm}{5mm}\rule{5mm}{0mm}}
