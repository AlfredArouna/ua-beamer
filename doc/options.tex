\section{Theme options}

The theme comes with several options, all of which can be given in a comma-separated list like in
\begin{lstlisting}
\usetheme[options]{UniversiteitAntwerpen}
\end{lstlisting}

\begin{center}
\begin{tabulary}{12cm}{p{3cm}L}\toprule
\lstinline!dark!
&
Just like the Universiteit Antwerpen itself provides two different flavors of PowerPoint\textsuperscript{\textregistered} presentations, one with a light background and one with a darker, this \texttt{beamer} themes inherits options for both. By default, the theme applies the light theme, this options switches to the dark counterpart. See figure \ref{fig:example1}.\\\midrule
\lstinline!darktitle!
&
With \lstinline!darktitle! (and not \lstinline!dark!), the title page will have a dark background while all other slides will retain the light background.\\\midrule
\lstinline!framenumber!
&
The \lstinline!framenumber! option makes sure that the number of the current frame is displayed. In the light scheme the current frame is displayed in the lower right corner of each frame, the dark scheme places them in the upper right corner.\\\midrule
\lstinline!totalframenumber!
&
With \lstinline!framenumber! turned on, the \lstinline!totalframenumber! option makes sure that the total number of frames is displayed alongside with the current frame number.\\\bottomrule
\end{tabulary}
\end{center}

\begin{note}
The \texttt{beamer} option \lstinline!compress! is respected in the sense that, if it is provided, header and footer will take less space such that there is more space for actual frame content.
\end{note}